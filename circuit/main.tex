\documentclass[journal,12pt,twocolumn]{IEEEtran}

\usepackage{enumitem}
\usepackage{amsmath}
\usepackage{amssymb}
\usepackage{gensymb}
\usepackage{graphicx}
\usepackage{txfonts}         
\usepackage{listings}
\usepackage{lstautogobble}
\usepackage{mathtools}
\usepackage{bm}
\usepackage{hyperref}
\usepackage{polynom}
\usepackage{siunitx}
\usepackage{verbatim} 
\usepackage[siunitx]{circuitikz}

\newcommand{\solution}{\noindent \textbf{Solution: }}
\providecommand{\pr}[1]{\ensuremath{\Pr\left(#1\right)}}
\providecommand{\brak}[1]{\ensuremath{\left(#1\right)}}
\providecommand{\cbrak}[1]{\ensuremath{\left\{#1\right\}}}
\providecommand{\sbrak}[1]{\ensuremath{\left[#1\right]}}
\providecommand{\mean}[1]{E\left[ #1 \right]}
\providecommand{\var}[1]{\mathrm{Var}\left[ #1 \right]}
\providecommand{\der}[1]{\mathrm{d} #1}
\providecommand{\gauss}[2]{\mathcal{N}\ensuremath{\left(#1,#2\right)}}
\providecommand{\mbf}{\mathbf}
\providecommand{\abs}[1]{\left\vert#1\right\vert}
\providecommand{\norm}[1]{\left\lVert#1\right\rVert}
\providecommand{\z}[1]{{\mathcal{Z}}\cbrak{#1}}
\providecommand{\ztrans}{\overset{\mathcal{Z}}{ \rightleftharpoons}}
\providecommand{\system}[1]{\overset{\mathcal{#1}}{ \longleftrightarrow}}
\providecommand{\parder}[2]{\frac{\partial}{\partial #2} \brak{#1}}

\let\StandardTheFigure\thefigure
\let\vec\mathbf

\numberwithin{equation}{section}
\numberwithin{figure}{section}
\renewcommand{\thefigure}{\theenumi}
\renewcommand\thesection{\arabic{section}}

\newcommand{\myvec}[1]{\ensuremath{\begin{pmatrix}#1\end{pmatrix}}}
\newcommand{\mymat}[1]{\ensuremath{\begin{bmatrix}#1\end{bmatrix}}}
\newcommand{\mydet}[1]{\ensuremath{\begin{vmatrix}#1\end{vmatrix}}}
\newcommand{\define}{\stackrel{\triangle}{=}}

\DeclareMathOperator*{\argmin}{arg\,min}
\DeclareMathOperator*{\argmax}{arg\,max}

\makeatletter
\def\pld@CF@loop#1+{%
    \ifx\relax#1\else
        \begingroup
          \pld@AccuSetX11%
          \def\pld@frac{{}{}}\let\pld@symbols\@empty\let\pld@vars\@empty
          \pld@false
          #1%
          \let\pld@temp\@empty
          \pld@AccuIfOne{}{\pld@AccuGet\pld@temp
                            \edef\pld@temp{\noexpand\pld@R\pld@temp}}%
           \pld@if \pld@Extend\pld@temp{\expandafter\pld@F\pld@frac}\fi
           \expandafter\pld@CF@loop@\pld@symbols\relax\@empty
           \expandafter\pld@CF@loop@\pld@vars\relax\@empty
           \ifx\@empty\pld@temp
               \def\pld@temp{\pld@R11}%
           \fi
          \global\let\@gtempa\pld@temp
        \endgroup
        \ifx\@empty\@gtempa\else
            \pld@ExtendPoly\pld@tempoly\@gtempa
        \fi
        \expandafter\pld@CF@loop
    \fi}
\def\pld@CMAddToTempoly{%
    \pld@AccuGet\pld@temp\edef\pld@temp{\noexpand\pld@R\pld@temp}%
    \pld@CondenseMonomials\pld@false\pld@symbols
    \ifx\pld@symbols\@empty \else
        \pld@ExtendPoly\pld@temp\pld@symbols
    \fi
    \ifx\pld@temp\@empty \else
        \pld@if
            \expandafter\pld@IfSum\expandafter{\pld@temp}%
                {\expandafter\def\expandafter\pld@temp\expandafter
                    {\expandafter\pld@F\expandafter{\pld@temp}{}}}%
                {}%
        \fi
        \pld@ExtendPoly\pld@tempoly\pld@temp
        \pld@Extend\pld@tempoly{\pld@monom}%
    \fi}
\makeatother

\lstset {
	frame=single, 
	breaklines=true,
	columns=fullflexible,
	autogobble=true
}             
                               
\title{Circuits and Transforms \\ \Large EE3900: Linear Systems and Signal Processing \\ \large Indian Institute of Technology Hyderabad}
\author{J Sai Sri Hari Vamshi \\ \normalsize AI21BTECH11014 \\ \vspace*{20pt} \normalsize 31 Oct 2022}


\begin{document}
	
	\maketitle
	
	\section{Definitions}
	\begin{enumerate}[label=\thesection.\arabic*,ref=\thesection.\theenumi]
	
		\item The unit step function is defined as
			\begin{align}
				u(t) =
				\begin{cases}
					1 & t > 0 \\
					\frac{1}{2} & t = 0 \\
					0 & t < 0
				\end{cases}
			\end{align}
		
		\item The Laplace transform of $g(t)$ is defined as 
			\begin{align}
				G(s) = \int_{-\infty}^{\infty} g(t) e^{-st}\, \der{t}
			\end{align}
	
	\end{enumerate}
	
	\section{Laplace Transform}
	\begin{enumerate}[label=\thesection.\arabic*.,ref=\thesection.\theenumi]
	
		\item In the circuit, the switch S is connected to position P for a long time so that the charge on the capacitor becomes $q_1$ \SI{}{\micro\coulomb}. Then S is switched to position Q.  After a long time, the charge on the capacitor is $q_2$ \SI{}{\micro\coulomb}\\
	
		\item Draw the circuit using latex-tikz
	
		\solution

			\begin{figure}[!ht]
			\centering
				\begin{circuitikz} \draw
					(0,3) to[battery1, l_=1<\volt>] (0,0) -- (8,0)
					(8,3) to[battery1, l=2<\volt>] (8,0)
					(4,0) to[C=1<\micro\farad>] (4,3)
						to[R=$2\,\Omega$] (8,3)
					(4,3) to[R, l_=$1\,\Omega$] (1,3)
						to[nos, l_=S, mirror] (0,3) node[label={left:P}]{}
					(0.7,0) -- (0.7,2.7) node[label={right:Q}]{}
					;
				\end{circuitikz}
				\caption{Circuit diagram of the circuit in question}
				\label{fig:ckt}
			\end{figure}

		\item Find $q_1$.

		\solution\\
			After a long time when a steady state is achieved, a capacitor behaves like an open circuit, that is, current passing through it is zero.\\
			\begin{figure}[!ht]
				\centering 
				\begin{circuitikz} \draw 
					(0,3) to[battery1, l =1<\volt>] (0,0) node[label={below:0}]{}
						-- (8,0) node[label={below:0}]{}
					(8,3) node[label={above:2}]{} to[battery1, l=2<\volt>] (8,0)
					(4,3) node[label={above:$V$}] {} to[R=$2\,\Omega$] (8,3)
					(4,3) to[R, l_=$1\,\Omega$] (0,3) node[label={above:1}]{}
					(4,0) node[ground]{}
					;
				\end{circuitikz}
				\caption{Circuit diagram at steady state before flipping the switch}
			\end{figure}

			By Kirchoff's Law we can get,

			\begin{align}
				\frac{V-1}{1} & - \frac{V-2}{2} = 0\\
				\implies & V = \SI[parse-numbers=false]{\frac{4}{3}}{\volt}\\
				\implies & q_1 = CV = \SI[parse-numbers=false]{\frac{4}{3}}{\micro\coulomb}
			\end{align}\ \\ \ \\ \ \\

		\item Show that the Laplace Transform of $u(t)$ is $\frac{1}{s}$ and find the ROC.\\

		\solution\\
			The Laplace Transform of $u(t)$ is given by,
			\begin{align}
				\mathcal{L}\cbrak{u(t)} & = \int_{-\infty}^\infty u(t) e^{-st} dt \\
							& = \int_0^\infty e^{-st} dt\\
							& = \lim_{R \to \infty} \frac{1-e^{-sR}}{s}
			\end{align}
			This limit is finite only if $\Re(s) > 0$, which is going to be its $ROC$.\\

			Therefore, 
			\begin{align}
				u(t) \system{L} \frac{1}{s} \qquad \Re(s) > 0 
			\end{align}

		\item Show that,
			\begin{align}
				e^{-at}u(t) \system{L} \frac{1}{s+a} \qquad a > 0
			\end{align}
			and find $ROC$.

		\solution\\
			The Laplace Transform of $e^{-at}u(t)$ for $a > 0$ is given by,
			\begin{align}
				\mathcal{L}\cbrak{u(t)} & = \int_{-\infty}^\infty e^{-at} u(t) e^{-st} dt \\
							& = \int_0^\infty e^{-(s+a)t} dt\\
							& = \lim_{R \to \infty} \frac{1-e^{-(s+a)R}}{s+a}
			\end{align}
			This limit is finite only if $\Re(s+a)>0$, which is going to be its ROC

			Therefore
			\begin{align}
				e^{-at}u(t)\system{L}\frac{1}{s+a}\qquad\Re(s)>-a
			\end{align}
			since $a$ is real.
		\item Now consider the following resistive circuit transformed from Fig. \ref{fig:ckt}\\
			\begin{figure}[!ht]
				\centering
				\begin{circuitikz} \draw
					(0,3) to[battery1, l_=$V_1(s)$] (0,0) node[label={below:0}]{}
						-- (6,0) node[label={below:0}]{}
					(6,3) node[label={above:$V_2$}]{} to[battery1, l_=$V_2(s)$] (6,0)
					(3,3) node[label={above:$V_c(s)$}] {} to[generic, l=$R_2$] (6,3)
					(3,3) to[generic, l_=$R_1$] (0,3) node[label={above:$V_1$}]{}
					(3,0) node[ground]{} to[generic, l=$\frac{1}{sC_0}$] (3,3)
					;
				\end{circuitikz}
				\caption{Circuit diagram in $s$-domain before flipping the switch}
				\label{fig:lap}
			\end{figure}
			where
			\begin{align}
				u(t) \system{L} V_1(s)\\
				2u(t) \system{L} V_2(s)
			\end{align}
			Find the voltage across the capacitor $V_c(s)$\\

		\solution\\
			\begin{align}
				V_1(s) & = \frac{1}{s}\qquad \Re(s) > 0\\
				V_2(s) & = \frac{2}{s}\qquad \Re(s) > 0\\
			\end{align}

			By Kirchoff's Junction Law, we get,

			\begin{align}
				\frac{V_c - V_1}{R_1} + \frac{V_c - V_2}{R_2} & +\frac{V_c - 0}{\frac{1}{sC_0}} = 0\\
				V_c\left( \frac{1}{R_1} + \frac{1}{R_2} + sC_0 \right) & = \frac{V_1}{R_1} + \frac{V_2}{R_2} \\
				V_c(s) & = \frac{\frac{1}{sR_1} + \frac{2}{sR_2}}{\frac{1}{R_1}+\frac{1}{R_2}+sC_0}\\
				V_c(s) & = \frac{\frac{1}{R_1C_0} + \frac{2}{R_2C_0}}{s\left( s + \frac{1}{R_1C_0} + \frac{1}{R_2C_0} \right)}
			\end{align}

		\item Find $v_c(t)$. Plot the graph using Python.\\

		\solution\\
			Upon performing the partial fraction decomposition upon the above obtained result, we get,
			\begin{align}
				V_c(s) & = \frac{\frac{1}{R_1C_0} + \frac{2}{R_2C_0}}{\frac{1}{R_1C_0} + \frac{1}{R_2C_0}}\left( \frac{1}{s} - \frac{1}{s} + \frac{1}{R_1C_0} + \frac{1}{R_2C_0} \right), \ \ \Re(s) > 0
			\end{align}
			On taking the inverse Laplace Transform we get,
			\begin{align}
				v_c(t) & = \frac{2R_1 + R_2}{R_1 + R_2}\left( u(t) - e^{-\left(\frac{1}{R_1} + \frac{1}{R_2}\right)\frac{1}{C_0}} \right)
			\end{align}
			
	\end{enumerate}
\end{document}
